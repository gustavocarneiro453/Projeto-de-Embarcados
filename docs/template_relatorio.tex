% Template de Relatório Técnico - ABNT2
% Projeto IoT - Estação Meteorológica com ESP32
% Sistemas Embarcados

\documentclass[12pt,a4paper]{article}
\usepackage[utf8]{inputenc}
\usepackage[brazil]{babel}
\usepackage[T1]{fontenc}
\usepackage{geometry}
\usepackage{graphicx}
\usepackage{float}
\usepackage{listings}
\usepackage{xcolor}
\usepackage{hyperref}
\usepackage{amsmath}
\usepackage{indentfirst}

% Configurações de página
\geometry{
    left=3cm,
    right=2cm,
    top=3cm,
    bottom=2cm
}

% Configurações de código
\lstset{
    language=C++,
    basicstyle=\ttfamily\small,
    keywordstyle=\color{blue},
    commentstyle=\color{green},
    stringstyle=\color{red},
    numbers=left,
    numberstyle=\tiny,
    stepnumber=1,
    numbersep=5pt,
    frame=single,
    breaklines=true,
    breakatwhitespace=true,
    tabsize=2
}

% Informações do documento
\title{Projeto IoT: Estação Meteorológica com ESP32}
\author{Grupo X\\[0.5cm]
        Nome do Aluno 1\\
        Nome do Aluno 2\\
        Nome do Aluno 3\\
        Nome do Aluno 4}
\date{\today}

\begin{document}

% Capa
\maketitle
\thispagestyle{empty}
\newpage

% Sumário
\tableofcontents
\newpage

% Introdução
\section{Introdução}
\label{sec:introducao}

\subsection{Contexto}
A Internet das Coisas (IoT) tem se tornado cada vez mais presente em nosso dia a dia, permitindo a conexão de dispositivos físicos à internet para coleta, processamento e visualização de dados em tempo real. Este projeto apresenta uma aplicação prática de IoT utilizando microcontroladores ESP32 para monitoramento ambiental.

\subsection{Motivação}
O monitoramento de parâmetros ambientais como temperatura e umidade é essencial em diversas aplicações, desde residências até ambientes industriais. A implementação de um sistema IoT permite o acompanhamento remoto desses parâmetros através de uma interface web, facilitando a tomada de decisões e o controle de ambientes.

\subsection{Objetivos}
\begin{itemize}
    \item Desenvolver um sistema IoT completo utilizando ESP32 e protocolo MQTT
    \item Implementar coleta de dados de temperatura e umidade através de sensores
    \item Criar um dashboard web para visualização em tempo real dos dados coletados
    \item Demonstrar a comunicação sem fio entre dispositivos embarcados e servidor
\end{itemize}

% Metodologia
\section{Metodologia}
\label{sec:metodologia}

\subsection{Diagrama do Sistema}
A Figura \ref{fig:diagrama} apresenta o diagrama de blocos do sistema completo.

\begin{figure}[H]
    \centering
    \includegraphics[width=0.9\textwidth]{imagens/diagrama_sistema.png}
    \caption{Diagrama de blocos do sistema}
    \label{fig:diagrama}
\end{figure}

\subsection{Lista de Hardware}
\begin{itemize}
    \item 1x Módulo ESP32 (NodeMCU)
    \item 1x Sensor DHT11 (temperatura e umidade)
    \item 1x Resistor 10k$\Omega$
    \item 1x Raspberry Pi (modelo 3B+ ou superior)
    \item Jumpers e protoboard
    \item Fonte de alimentação USB para ESP32
\end{itemize}

\subsection{Lista de Software}
\begin{itemize}
    \item \textbf{Raspberry Pi:}
    \begin{itemize}
        \item Raspberry Pi OS (Linux)
        \item Mosquitto MQTT Broker
        \item Python 3.x
        \item Flask (framework web)
        \item paho-mqtt (biblioteca MQTT)
    \end{itemize}
    \item \textbf{ESP32:}
    \begin{itemize}
        \item Arduino IDE ou PlatformIO
        \item Biblioteca PubSubClient (MQTT)
        \item Biblioteca DHT sensor library
        \item FreeRTOS (já incluído no ESP32)
    \end{itemize}
\end{itemize}

\subsection{Fluxo de Comunicação}
O sistema utiliza o protocolo MQTT sobre Wi-Fi para comunicação entre os dispositivos:

\begin{enumerate}
    \item O ESP32 conecta-se à rede Wi-Fi local
    \item O sensor DHT11 é lido periodicamente (a cada 5 segundos)
    \item Os dados são publicados no broker MQTT (Raspberry Pi) nos tópicos:
    \begin{itemize}
        \item \texttt{sensor/temperature}
        \item \texttt{sensor/humidity}
        \item \texttt{sensor/status}
    \end{itemize}
    \item O dashboard web (Flask) subscreve aos tópicos MQTT
    \item Os dados são exibidos em tempo real na interface web
\end{enumerate}

% Resultados
\section{Resultados}
\label{sec:resultados}

\subsection{Dashboard}
A Figura \ref{fig:dashboard} apresenta a interface do dashboard web desenvolvido.

\begin{figure}[H]
    \centering
    \includegraphics[width=0.9\textwidth]{imagens/dashboard.png}
    \caption{Interface do dashboard web}
    \label{fig:dashboard}
\end{figure}

\subsection{Dados Coletados}
Os dados coletados são exibidos em tempo real através de gráficos interativos, permitindo a visualização do histórico de temperatura e umidade.

\subsection{Evidências de Funcionamento}
O sistema foi testado e validado, demonstrando:
\begin{itemize}
    \item Conexão estável entre ESP32 e broker MQTT
    \item Coleta precisa de dados do sensor DHT11
    \item Visualização em tempo real no dashboard web
    \item Reconexão automática em caso de falhas de conexão
\end{itemize}

% Conclusão
\section{Conclusão}
\label{sec:conclusao}

\subsection{Desafios Enfrentados}
Durante o desenvolvimento do projeto, foram enfrentados os seguintes desafios:
\begin{itemize}
    \item Configuração inicial do broker MQTT no Raspberry Pi
    \item Sincronização entre múltiplos dispositivos ESP32
    \item Otimização da interface web para atualizações em tempo real
\end{itemize}

\subsection{Aprendizados}
O projeto proporcionou aprendizado significativo em:
\begin{itemize}
    \item Programação de microcontroladores ESP32
    \item Protocolo MQTT e sua implementação prática
    \item Desenvolvimento de aplicações web com Flask
    \item Integração de sistemas embarcados com servidores web
\end{itemize}

\subsection{Melhorias Futuras}
Como melhorias futuras, sugere-se:
\begin{itemize}
    \item Implementação de autenticação MQTT (TLS/SSL)
    \item Armazenamento de dados em banco de dados (PostgreSQL/MySQL)
    \item Adição de mais sensores (pressão, qualidade do ar)
    \item Desenvolvimento de aplicativo mobile
    \item Implementação de alertas por e-mail/SMS
    \item Suporte a IPv6
\end{itemize}

% Apêndices
\appendix
\section{Código Fonte - ESP32}
\label{ap:codigo_esp32}

O código completo do firmware ESP32 está disponível no repositório GitHub do projeto.

\section{Configuração do Broker MQTT}
\label{ap:config_mqtt}

A configuração do Mosquitto MQTT Broker está disponível no diretório \texttt{raspberry-pi/broker/} do repositório.

% Referências
\begin{thebibliography}{9}
\bibitem{esp32}
Espressif Systems. \textit{ESP32 Technical Reference Manual}. Disponível em: \url{https://www.espressif.com/sites/default/files/documentation/esp32_technical_reference_manual_en.pdf}

\bibitem{mqtt}
OASIS. \textit{MQTT Version 3.1.1}. Disponível em: \url{https://docs.oasis-open.org/mqtt/mqtt/v3.1.1/mqtt-v3.1.1.html}

\bibitem{flask}
Flask Documentation. \textit{Flask: Web Development, One Drop at a Time}. Disponível em: \url{https://flask.palletsprojects.com/}

\bibitem{dht11}
DHT11 Datasheet. \textit{DHT11 Humidity \& Temperature Sensor}. Disponível em: \url{https://www.mouser.com/datasheet/2/758/DHT11-Technical-Data-Sheet-Translated-Version-1143054.pdf}
\end{thebibliography}

\end{document}

